\documentclass[12pt]{article}
\usepackage{geometry}
\newgeometry{tmargin=2cm, bmargin=2cm, lmargin=2cm, rmargin=2cm}
\linespread{1.75} 
\usepackage[utf8]{inputenc}
\usepackage{polski}
\usepackage{amsmath}
\usepackage{amsthm}
\usepackage{graphicx}
\usepackage{subcaption}
\usepackage{caption}
\usepackage{amssymb}
\usepackage{amsfonts}
\usepackage{xcolor}
\usepackage[export]{adjustbox}
\usepackage{mathtools}
\usepackage[document]{ragged2e}
\usepackage{float}
\usepackage{eucal}
\usepackage{titlesec}
\usepackage{scrextend}
\usepackage{tikz}
\usepackage{subfiles}
\usepackage{siunitx}
\usepackage{listings}
\usepackage{xcolor}
\usepackage{pgfplots}
\usetikzlibrary{calc,patterns,angles,quotes}
\everymath{\displaystyle}

\title{Dokumentacja gry w statki}
\author{Maciej Ziobro}
\date{\today}


\lstset{
    basicstyle=\ttfamily,
    backgroundcolor=\color{white},
    frame=single,
    breaklines=true,
    postbreak=\mbox{\textcolor{red}{$\hookrightarrow$}\space},
    numbers=left,
    numberstyle=\tiny\color{gray},
    language=Python,
}

\begin{document}

\maketitle
\newpage

\section{Wstęp}
Ten dokument zawiera dokumentację dla gry w statki zaimplementowanej w języku Python przy użyciu biblioteki Pygame.

\section{Przegląd Kodu}
Gra składa się z kilku modułów, w tym ship.py, ppoints.py i bot.py. Główna logika gry jest obecna w dostarczonym skrypcie Pythona.

\section{Funkcjonalności}
\subsection{Rozmieszczenie Statków}
Funkcja \texttt{ships\_placement} generuje początkowe rozmieszczenie statków na planszy gry. Przyjmuje listę rozmiarów statków jako argument i zwraca listę obiektów statków.

\subsection{Graficzny Interfejs Użytkownika (GUI)}
Gra korzysta z Pygame do interfejsu graficznego. Inicjalizuje środowisko Pygame, definiuje kolory i obsługuje zdarzenia, takie jak zakończenie gry.

\subsection{Główna Pętla Gry}
Główna pętla gry kontroluje przebieg gry. Obsługuje ruchy gracza i bota, aktualizuje wyświetlanie i sprawdza zakończenie gry.

\subsection{Menu}
Gra zawiera menu startowe i końcowe. Funkcja \texttt{start\_menu} pozwala graczowi wybrać, kto rozpocznie grę, natomiast funkcja \texttt{end\_menu} prosi gracza o ponowną grę lub wyjście.

\section{Struktura Kodu}
Kod jest zorganizowany na logiczne sekcje, takie jak rozmieszczenie statków, GUI, menu i główna pętla gry. Każda sekcja jest czytelnie skomentowana, co ułatwia czytanie kodu.

\section{Zależności}
Gra polega na bibliotece Pygame do renderowania grafiki. Upewnij się, że Pygame jest zainstalowane przed uruchomieniem kodu.

\section{Podsumowanie}
Dostarczony skrypt w Pythonie implementuje prostą grę w statki z graficznym interfejsem użytkownika. Oferuje przyjemne doświadczenie z połączeniem strategicznego rozmieszczenia statków i turowej walki.

\end{document}
